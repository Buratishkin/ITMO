%починить угол, вставить угол
\thispagestyle{empty} % нет номеров страниц
\documentclass[14pt,a4paper]{report}
\usepackage[T1,T2A]{fontenc}
\usepackage[utf8x]{inputenc}
\usepackage[10pt]{extsizes}
\usepackage{geometry}
\usepackage{xcolor}
\usepackage{tikz}
\usepackage{graphicx, caption}
\usepackage{wrapfig}
\usepackage{amssymb}
\usepackage{geometry}

    
\geometry{top=4em,right=2em,left=2em,bottom=4em}

\begin{document}
\newcommand{\Mainclt}{\raisebox{0pt}[\headheight][30pt]{\vbox{\hbox to\textwidth{74\hfilКНИГА I ПРЕДЛ. XLVIII. ТЕОРЕМА\hfil}}}}

\begin{minipage}[t]{0.3\textwidth}
    \hfill \\
    \begin{center}
    \begin{tikzpicture}
        \draw[red, fill=red] (0, -4) --  (0.6,-4) arc(0:90:0.6) -- cycle;
        \draw[yellow, fill=yellow] (0, -4) -- (0, -3.4) arc(90:180:0.6) -- cycle;
    
        \draw[dashed, black, ultra thick] (-2,-4) --  (0,-4);
        \draw[black, ultra thick] (0,-4) -- (2,-4);
        \draw[blue, ultra thick] (0,-4) -- (0,-1);
        \draw[red, ultra thick] (0,-1) -- (2,-4);
        \draw[dashed, red, ultra thick] (0,-1) -- (-2,-4);
    
        \node[below] at (-2.2, -4) {\tiny$D$};
        \node[below] at (0, -4) {\tiny$A$};
        \node[below] at (2.2, -4) {\tiny$C$};
        \node[above] at (0, -1) {\tiny$B$};
    \end{tikzpicture}
    \end{center}
\end{minipage}\hfill
\begin{minipage}[t]{0.57\textwidth}
    \Mainclt
    \begin{wrapfigure}{l}{0.2\linewidth}
        \includegraphics[width=0.9\linewidth]{буква.jpg}
    \end{wrapfigure}\\
    \slshape сли \textit{в треугольнике квадрат одной стороны}
    \begin{tikzpicture}
        \draw[blue,ultra thick] (0,0) -- (1,0);
        \node[above] at (0,0) {\tiny B};
        \node[above] at (1,0) {\tiny C};
    \end{tikzpicture}
    \textit{равен сумме квадратов двух других сторон}
    \begin{tikzpicture}
        \draw[blue,ultra thick] (0,0) -- (1,0);
        \node[above] at (0,0) {\tiny A};
        \node[above] at (1,0) {\tiny B};
    \end{tikzpicture} 
    \textit{и}
    \begin{tikzpicture}
        \draw[black,ultra thick] (0,0) -- (1,0);
        \node[above] at (0,0) {\tiny A};
        \node[above] at (1,0) {\tiny C};
    \end{tikzpicture},
    \textit{то угол}
    \begin{tikzpicture}
        \raisebox{-1 em}{
            \draw[red, fill=red] (0, -4) --  (0.6,-4) arc(0:90:0.6) -- cycle;
            \node[above] at (0, -3.4) {\tiny B};
            \node[below] at (-0.1, -3.9) {\tiny A};
            \node[right] at (0.6, -4) {\tiny C};
        }
    \end{tikzpicture}
    \textit{заключенный между этими двумя сторонами прямой.}

    \upshape

    \begin{center}
        Проведем 
        \begin{tikzpicture}
            \draw[dashed,black,ultra thick] (0,0) -- (1,0);
            \node[above] at (0,0) {\tiny A};
            \node[above] at (1,0) {\tiny D};
        \end{tikzpicture}
        перпендикуляр
        \begin{tikzpicture}
            \draw[blue,ultra thick] (0,0) -- (1,0);
            \node[above] at (0,0) {\tiny A};
            \node[above] at (1,0) {\tiny B};
        \end{tikzpicture} \\
        и = 
        \begin{tikzpicture}
            \draw[black,ultra thick] (0,0) -- (1,0);
            \node[above] at (0,0) {\tiny A};
            \node[above] at (1,0) {\tiny C};
        \end{tikzpicture}
        (пр. I. \unskip
        \begin{scriptsize}
            II
        \end{scriptsize},I.\raisebox{-0.2em}{3}) \\
        также проведем 
        \begin{tikzpicture}
            \draw[dashed,red,ultra thick] (0,0) -- (1,0);
            \node[above] at (0,0) {\tiny B};
            \node[above] at (1,0) {\tiny D};
        \end{tikzpicture}. \\
        Поскольку
        \begin{tikzpicture}
            \draw[dashed,black,ultra thick] (0,0) -- (1,0);
            \node[above] at (0,0) {\tiny A};
            \node[above] at (1,0) {\tiny D};
        \end{tikzpicture}
        =
        \begin{tikzpicture}
            \draw[black,ultra thick] (0,0) -- (1,0);
            \node[above] at (0,0) {\tiny A};
            \node[above] at (1,0) {\tiny C};
        \end{tikzpicture}
        (постр.) \\
        \begin{tikzpicture}
            \draw[dashed,black,ultra thick] (0,0) -- (1,0);
            \node[above] at (0,0) {\tiny A};
            \node[above] at (1,0) {\tiny D\raisebox{+0.6em}{2}};
        \end{tikzpicture}  
        =
        \begin{tikzpicture}
            \draw[black,ultra thick] (0,0) -- (1,0);
            \node[above] at (0,0) {\tiny A};
            \node[above] at (1,0) {\tiny C\raisebox{+0.6em}{2}};
        \end{tikzpicture}; \\
        $\therefore$ 
        \begin{tikzpicture}
            \draw[dashed,black,ultra thick] (0,0) -- (1,0);
            \node[above] at (0,0) {\tiny A};
            \node[above] at (1,0) {\tiny D\raisebox{+0.6em}{2}};
        \end{tikzpicture}  
        + 
        \begin{tikzpicture}
            \draw[blue,ultra thick] (0,0) -- (1,0);
            \node[above] at (0,0) {\tiny A};
            \node[above] at (1,0) {\tiny B\raisebox{+0.6em}{2}};
        \end{tikzpicture}
        =
        \begin{tikzpicture}
            \draw[black,ultra thick] (0,0) -- (1,0);
            \node[above] at (0,0) {\tiny A};
            \node[above] at (1,0) {\tiny C\raisebox{+0.6em}{2}};
        \end{tikzpicture};
        +
        \begin{tikzpicture}
            \draw[blue,ultra thick] (0,0) -- (1,0);
            \node[above] at (0,0) {\tiny A};
            \node[above] at (1,0) {\tiny B\raisebox{+0.6em}{2}};
        \end{tikzpicture} \\
        но
        \begin{tikzpicture}
            \draw[dashed,black,ultra thick] (0,0) -- (1,0);
            \node[above] at (0,0) {\tiny A};
            \node[above] at (1,0) {\tiny D\raisebox{+0.6em}{2}};
        \end{tikzpicture}  
        +
        \begin{tikzpicture}
            \draw[blue,ultra thick] (0,0) -- (1,0);
            \node[above] at (0,0) {\tiny A};
            \node[above] at (1,0) {\tiny B\raisebox{+0.6em}{2}};
        \end{tikzpicture}
        =
        \begin{tikzpicture}
            \draw[dashed,red,ultra thick] (0,0) -- (1,0);
            \node[above] at (0,0) {\tiny B};
            \node[above] at (1,0) {\tiny D};
        \end{tikzpicture}
        (пр. I.\raisebox{-0.2em}{47}), \\
        и   
        \begin{tikzpicture}
            \draw[black,ultra thick] (0,0) -- (1,0);
            \node[above] at (0,0) {\tiny A};
            \node[above] at (1,0) {\tiny C\raisebox{+0.6em}{2}};
        \end{tikzpicture};
        +
        \begin{tikzpicture}
            \draw[blue,ultra thick] (0,0) -- (1,0);
            \node[above] at (0,0) {\tiny A};
            \node[above] at (1,0) {\tiny B\raisebox{+0.6em}{2}};
        \end{tikzpicture}
        = 
        \begin{tikzpicture}
            \draw[red,ultra thick] (0,0) -- (1,0);
            \node[above] at (0,0) {\tiny B};
            \node[above] at (1,0) {\tiny C\raisebox{+0.6em}{2}};
        \end{tikzpicture}
        (гип.) \\
        $\therefore$
        \begin{tikzpicture}
            \draw[dashed,red,ultra thick] (0,0) -- (1,0);
            \node[above] at (0,0) {\tiny B};
            \node[above] at (1,0) {\tiny D\raisebox{+0.6em}{2}};
        \end{tikzpicture}
        =
        \begin{tikzpicture}
            \draw[red,ultra thick] (0,0) -- (1,0);
            \node[above] at (0,0) {\tiny B};
            \node[above] at (1,0) {\tiny C\raisebox{+0.6em}{2}};
        \end{tikzpicture}, \\
        $\therefore$
        \begin{tikzpicture}
            \draw[dashed,red,ultra thick] (0,0) -- (1,0);
            \node[above] at (0,0) {\tiny B};
            \node[above] at (1,0) {\tiny D};
        \end{tikzpicture}
        =
        \begin{tikzpicture}
            \draw[red,ultra thick] (0,0) -- (1,0);
            \node[above] at (0,0) {\tiny B};
            \node[above] at (1,0) {\tiny C};
        \end{tikzpicture}; \\
        и $\therefore$
        \begin{tikzpicture}
            \raisebox{-1.4em}{
                \draw[yellow, fill=yellow] (0, -4) -- (0, -3.4) arc(90:180:0.6) -- cycle;
                \node[above] at (0, -3.4) {\tiny B};
                \node[below] at (0.1, -3.9) {\tiny A};
                \node[left] at (-0.6, -4){\tiny D};
            }
        \end{tikzpicture}
        =
        \begin{tikzpicture}
            \raisebox{-1.4em}{
                \draw[red, fill=red] (0, -4) --  (0.6,-4) arc(0:90:0.6) -- cycle;
                \node[above] at (0, -3.4) {\tiny B};
                \node[below] at (-0.1, -3.9) {\tiny A};
                \node[right] at (0.6, -4){\tiny C};
            }
        \end{tikzpicture}
        (пр. I.8), \\
        следовательно
        \begin{tikzpicture}
            \raisebox{-1.4em}{
                \draw[red, fill=red] (0, -4) --  (0.6,-4) arc(0:90:0.6) -- cycle;
                \node[above] at (0, -3.4) {\tiny B};
                \node[below] at (-0.1, -3.9) {\tiny A};
                \node[right] at (0.6, -4){\tiny C};
            }
        \end{tikzpicture}
        прямой угол
    \end{center}

    \begin{flushright}
        ч. т. д.
    \end{flushright}
    
\end{minipage}

\end{document}